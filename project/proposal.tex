\documentclass {article}
\usepackage{fullpage}

\begin{document}

~\vfill
\begin{center}
\Large

A5 Project Proposal

Title:

Name: Jacob Bush

Student ID: 20558637

User ID: jbush
\end{center}
\vfill ~\vfill~
\newpage
\noindent{\Large \bf Final Project:}
\begin{description}
\item[Purpose]:

    To create an image that can simulate glass in a near-photorealistic manner.

\item[Statement]:

	For Ray Tracers: Paragraph describing interesting scene to be
		rendered and what features are needed to achieve
		this scene.

	Paragraph: What it's about.

	Paragraph: What to do.

	Paragraph: Why it is interesting and challenging.

	Paragraph: What I will learn

\item[Technical Outline]:
    % Basically, your objectives in your objective list should be fairly
    % short statements of the objective; you should provide additional
    % details about your objectives in this section to clarify what you
    % plan to do.

    %  Further, survey the important data structures and algorithms that
    %  will be necessary to achieve the goals, and (for ray tracing
    %  projects) lists the new commands
    %  that will need to be added to the input language.

    %  To  get  bold face: {\bf bold face words}.  To get italics: {\it italic
    %  face words}.  To  get typewriter font: {\tt typed words}.  To get
    %  larger  words:  {\large large  words}.   To  get smaller words: 
    %  {\small small words}.  
    
    For new primitives, I will implement cylinders and toruses. I will use the equations provided in class, and also by referencing Watt \& Watt \cite{WattQuad} and Wolfram Mathworld \cite{Weisstein}. This will require the creation of two new commands to the input language. The command  \texttt{gr.cylinder} will create a cylinder, and the command \texttt{gr.torus} will create a torus.
    Non-hierarchal versions of these commands (\texttt{gr.nh\_cylinder}, \texttt{gr.nh\_torus}) may also be added for convenience.

    For both texture and bump mapping, I will rely heavily on Watt \& Watt \cite{WattMapping}, as well as referencing Blinn \& Newell's 1976 paper \cite{Blinn} and Catmull's original dissertation on the topic \cite{Catmull}. Two new commands will be added to the input language to implement texture mapping. The command \texttt{gr.texturemap} will return a TextureMap object (by, for instance, reading an image file into a matrix of size equal to the image resolution). The object method \texttt{geonode:set\_texturemap} will apply the given texture map to the GeometryNode. We will similarily define two new commands for bump mapping: \texttt{gr.bumpmap} to create a BumpMap object, and \texttt{geonode:set\_bumpmap} to apply the bump map to a the GeometryNode.
    
    Reflection.
    
    Refraction.
    
    Caustics.
    
    Acceleration will be acieved through the use of octtrees. The space will be divided into octants, (eight regions) with each octant containing $0$ or more objects. Octants with more than a certain threshold of objects in them will be further subdivided. Instead of testing a ray's intersection against all objects in the scene, we test it against only the objects in the regions that the ray will pass through, reducting computational overhead \cite{WattOct}.

    Antialiasing will be achieved via the jittering method described in class. That is: subdivide each pixel in the scene into an $N\times N$ grid, for some fixed $N$. For each region in the subdivided grid, randomly select a point in that region, and trace a ray through that point. The resulting color of the pixel is the average of the colors returned by the $N^2$ rays.
    
    Depth of field.
    
    Final Scene.

\item[Bibliography]:

\begingroup
\renewcommand{\section}[2]{}%
\begin{thebibliography}{9}

%https://graphics.stanford.edu/courses/cs348b-05/readings.html

% \bibitem{Wang}
% Wang, Huamin. "Texture Mapping". department of Computer Science and Engineering. Ohio State University. \\\texttt{http://web.cse.ohio-state.edu/\textasciitilde wang.3602/courses/cse5542-2013-spring/15-texture.pdf}

\bibitem{WattQuad} 
Watt, Alan H.\& Watt, Mark (1992). Instersections: ray/quadratics. In Advanced Animation and Rendering Techniques: Theory and Practice. Addison-Wesley Professional. pp. 226-227.

\bibitem{Weisstein} 
Weisstein, Eric W. "Torus." From MathWorld--A Wolfram Web Resource.\\ \texttt{http://mathworld.wolfram.com/Torus.html}

\bibitem{WattMapping} 
Watt, Alan H.\& Watt, Mark (1992). Mapping techniques: texture and environment mapping. In Advanced Animation and Rendering Techniques: Theory and Practice. Addison-Wesley Professional. pp. 178-201.

\bibitem{Blinn}
Blinn, J. F., \& Newell, M. E. (1976). Texture and reflection in computer generated images. Communications of the ACM, 19(10), 542-547. \texttt{https://doi.org/10.1145/360349.360353}

\bibitem{Catmull}
Catmull, E. (1974). A subdivision algorithm for computer display of curved surfaces (PhD thesis). University of Utah.

\bibitem{WattOct} 
Watt, Alan H.\& Watt, Mark (1992). Spatial coherence. In Advanced Animation and Rendering Techniques: Theory and Practice. Addison-Wesley Professional. pp. 241-248.

% \bebitem{stanfordaccel}
% https://graphics.stanford.edu/courses/cs348b-05/lectures/lecture3/raytrace_ii.pdf

% http://www.pixartouchbook.com/storage/catmull_thesis.pdf

% Don't want to use this unless I have a non-draft version - pp might be different
% \bibitem{Pharr}
% Pharr, M., \& Humphreys, G. (2004). Physically based rendering: From theory to implementation. Morgan Kaufmann. pp. abc-abc

\end{thebibliography}
\endgroup

\end{description}
\newpage


\noindent{\Large\bf Objectives:}

{\hfill{\bf Full UserID:\hspace{0.2in}\underline{\hspace{0.5in}jbush\hspace{0.5in}}}\hfill{\bf Student ID:\hspace{0.2in}\underline{\hspace{0.4in}20558637\hspace{0.4in}}}\hfill}

\begin{enumerate}
     \item Primitives
     \item Texture Mapping
     \item Bump Mapping
     \item Reflection
     \item Refraction
     \item Caustics via ...
     \item Acceleration via Oct-tree
     \item Antialiasing
     \item Depth of Field
     \item Final Scene
\end{enumerate}

% Soft Shadows? Constructive solid geometry? glossy reflection / transmission? Radiosity (might be hard)? Path tracing? Depth of field?

% Delete % at start of next line if this is a ray tracing project
A4 extra objective: I did antialiasing as my extra objective for A4. \textbf{However}, I do not expect to receive the mark for it. If I \textit{do} recieve credit for this objective in A4, I will choose a new objective for the projcet.
\end{document}
